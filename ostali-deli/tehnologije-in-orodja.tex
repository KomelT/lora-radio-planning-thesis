\chapter{Uporabljene tehnologije in orodja}

Pri razvoju diplomske naloge so bila uporabljena sodobna orodja in tehnologije. Poleg tega so izbrane rešitve preverjene, odprtokodne ter široko uporabljane v industriji, kar zagotavlja stabilnost in dolgoročno vzdržnost sistema. V nadaljevanju so predstavljene ključne tehnologije in orodja, ki so bila uporabljena pri razvoju aplikacije, ter njihova vloga pri delovanju celotne rešitve.

\section{Vue.js}
Vue.js~\cite{vuejs} je odprtokodno ogrodje za gradnjo uporabniških vmesnikov, ki temelji na programskem jeziku JavaScript. Namenjen je gradnji enostranskih aplikacij z več manjšimi gradniki, ki jih je mogoče ponovno uporabiti na več različnih delih aplikacije. S tem se prihrani čas pri programiranju in zmanjša število vrstic kode.

Prva različica je ugledala luč sveta leta 2014. Zasnoval in ustvaril ga je Evan You kot izboljšavo ogrodja AngularJS~\cite{angularjs}, potem ko je z njim delal v podjetju Google.

\section{Tailwind CSS}
Tailwind CSS je sodobna knjižnica za oblikovanje, ki temelji na pristopu oblikovanja z razredi. Namesto pisanja lastnih slogov razvijalci uporabljajo vnaprej pripravljene razrede, kot so \texttt{ml-2}, \texttt{bg-orange-400} ali \texttt{block}, kar omogoča hitrejše prototipiranje in konsistenten dizajn. Ena od njegovih prednosti je tudi optimizacija za produkcijo, saj lahko odstrani neuporabljene razrede in s tem zmanjša velikost končne aplikacije.

\section{Python}
Python~\cite{python} je visokonivojski, interpretirani programski jezik, ki se odlikuje po enostavni in berljivi sintaksi. Njegove prednosti so obsežen standardni knjižnični nabor, aktivna skupnost ter široka podpora za različna področja – od spletnega razvoja in avtomatizacije do podatkovne znanosti in umetne inteligence. Python podpira več programerskih paradigem, vključno s postopkovnim, objektno usmerjenim in funkcijskim programiranjem. Zaradi enostavnega učenja in zmogljivosti je danes eden najbolj priljubljenih jezikov na svetu.

\section{SPLAT!}
SPLAT!~\cite{splatTool} (angl. \emph{Signal Propagation, Loss, And Terrain analysis tool}) je odprtokodno terminalsko orodje za analizo in simulacijo širjenja radijskega signala. Jedro orodja temelji na modelu Longley--Rice, znanem kot \emph{Irregular Terrain Model} (ITM), ter njegovi izboljšani različici \emph{Irregular Terrain with Obstructions Model} (ITWOM), ki omogočata napoved izgub poti in jakosti sprejetega signala ob upoštevanju razgibanega terena. SPLAT! pri izračunih uporablja digitalne modele višin, Fresnelove cone ter geometrijo vidne linije, kar omogoča realistično oceno pokritosti na frekvencah med 20~MHz in 20~GHz.

\section{Redis}
Redis~\cite{redis} je odprtokodna podatkovna baza v pomnilniku, ki uporablja model ključ-vrednost. Zasnovan je za hitro obdelavo podatkov z nizko zakasnitvijo in podpira različne podatkovne strukture, kot so seznami, množice, urejeni seznami in zgoščene tabele. Pogosto se uporablja kot predpomnilnik, sistem za sporočila ali podatkovna baza za časovno občutljive podatke.

\section{GeoServer}
GeoServer~\cite{geoserver} je odprtokodni strežnik, namenjen objavi in obdelavi prostorskih podatkov. Podpira standarde OGC (Open Geospatial Consortium), kot so WMS (Web Map Service), WFS (Web Feature Service) in WCS (Web Coverage Service). GeoServer omogoča dostop do prostorskih podatkov iz različnih virov (npr. shapefile, GeoTIFF, PostGIS) ter njihovo vizualizacijo in analizo. Zaradi svoje razširljivosti in združljivosti s številnimi GIS orodji je postal standardna rešitev za objavo prostorskih podatkov na spletu.

\section{Nginx}
Nginx~\cite{nginx} je odprtokodni spletni strežnik in posredniški strežnik (angl. \emph{reverse proxy}), ki se uporablja za strežbo statičnih vsebin, obdelavo zahtev HTTPS ter uravnoteženje obremenitve med strežniki. Zaradi svoje učinkovitosti, majhne porabe virov in zmožnosti hkratnega obravnavanja velikega števila povezav je postal ena najbolj priljubljenih rešitev za gostovanje spletnih aplikacij. Nginx podpira tudi modul za predpomnjenje in varnostne nastavitve, kar ga uvršča med ključne gradnike sodobne strežniške infrastrukture.

\section{Docker}
Docker~\cite{docker} je platforma za kontejnerizacijo, ki omogoča zagon aplikacij v izoliranih okoljih. Kontejnerji združujejo programsko opremo z vsemi njenimi odvisnostmi, kar omogoča enostaven prenos in reproducibilno izvajanje na različnih sistemih. Docker je postal standard v industriji, saj poenostavlja razvoj, testiranje in uvajanje aplikacij. Njegove prednosti so hitrost zagona, konsistenca med okolji in podpora za mikroservisno arhitekturo.
