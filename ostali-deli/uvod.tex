\chapter{Uvod}
Z razvojem tehnologij interneta stvari (IoT) se povečuje potreba po brezžičnih komunikacijskih rešitvah, ki omogočajo zanesljiv prenos podatkov na velikih razdaljah ob nizki porabi energije. Ena izmed takšnih tehnologij je LoRa, ki temelji na modulaciji razširjenega spektra s frekvenčnimi čirpi (angl. \emph{Chirp Spread Spectrum}) in omogoča robustno komunikacijo tudi v zahtevnih pogojih. Zaradi visoke občutljivosti sprejemnikov in prilagodljivih prenosnih parametrov se LoRa pogosto uporablja v okoljih, kot so pametno kmetijstvo, spremljanje okolja in pametna mesta, kjer so razdalje med napravami velike, infrastruktura pa omejena.~\cite{app9224753}

Gradnja LoRa omrežij pa s seboj prinaša izzive. Ključno vprašanje pri postavitvi tovrstnih omrežij je, kako pravilno načrtovati lokacije oddajnikov in sprejemnikov, da bo zagotovljena ustrezna pokritost, stabilnost in zanesljivost komunikacije. Na razpoložljivost povezave vplivajo številni dejavniki, kot so razgibanost terena, prisotnost fizičnih ovir, vremenske razmere ter izbrani tehnični parametri naprav. Zato je za učinkovito načrtovanje omrežja potrebno uporabiti ustrezna orodja za modeliranje širjenja radijskega signala.

Na trgu sicer že obstajajo nekatera orodja, kot so \texttt{GRASS-RaPlat}~\cite{raplatGrass}, \texttt{Radio Mobile Online}~\cite{ve2dbeRFPathCalc}, \texttt{Meshtastic Site Planner}~\cite{meshtasticSitePlanner} ali \texttt{SCADACore RF Line-of-Sight}~\cite{scadacoreRFLoS}, ki omogočajo simulacijo pokritosti ali analizo vidnega polja. Vendar se ta orodja pogosto izkažejo za neprilagodljiva, zaprta ali pa pokrivajo le posamezne vidike radijskega načrtovanja, kot sta simulacija pokritosti in analiza vidne linije. Uporabnik mora za celovit pregled običajno kombinirati več različnih rešitev, kar otežuje uporabo in podaljšuje proces načrtovanja.

Motivacija za to diplomsko nalogo izhaja iz potrebe po odprtokodnem, enostavnem in razširljivem orodju, ki bi združevalo različne vidike radijskega načrtovanja v eni sami aplikaciji. Cilj je razviti rešitev, ki bo uporabniku omogočala:
\begin{itemize}
    \item Simulacijo širjenja radijskega signala glede na tehnične parametre oddajnika in topografijo terena.
    \item Analizo vidne linije med izbranima točkama.
    \item Iskanje optimalnih lokacij oddajnikov glede na sprejemnike, kjer je nabor možnih lokacij oddajnikov določen s seznamom koordinat ali z izbranim geografskim območjem.
    \item Vizualizacijo rezultatov na interaktivnem zemljevidu, ki omogoča lažje razumevanje in podporo odločanju.
\end{itemize}

Glavni prispevki diplomske naloge so:
\begin{itemize}
    \item Odprtokodna spletna aplikacija \texttt{RF Site Planner}.
    \item Analiza odprtokodnega terminalskega orodja SPLAT! za radijsko načrtovanje.
    \item Eksperimentalna evalvacija orodja SPLAT! in razvite rešitve.
\end{itemize}
